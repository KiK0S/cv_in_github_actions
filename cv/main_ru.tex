%%%%%%%%%%%%%%%%%%%%%%%%%%%%%%%%%%%%%%%%%
% Medium Length Professional CV
% LaTeX Template
% Version 2.0 (8/5/13)
%
% This template has been downloaded from:
% http://www.LaTeXTemplates.com
%
% Original author:
% Rishi Shah 
%
% Important note:
% This template requires the resume.cls file to be in the same directory as the
% .tex file. The resume.cls file provides the resume style used for structuring the
% document.
%
%%%%%%%%%%%%%%%%%%%%%%%%%%%%%%%%%%%%%%%%%

%----------------------------------------------------------------------------------------
%	PACKAGES AND OTHER DOCUMENT CONFIGURATIONS
%----------------------------------------------------------------------------------------

\documentclass{resume} % Use the custom resume.cls style

\usepackage[left=0.75in,top=0.6in,right=0.75in,bottom=0.6in]{geometry} % Document margins
\usepackage{hyperref}
\hypersetup{
    colorlinks=true,
    linkcolor=blue,
    filecolor=magenta, 
    urlcolor=blue,
}
\usepackage[utf8]{inputenc}
\usepackage[russian]{babel}
\newcommand{\tab}[1]{\hspace{.2667\textwidth}\rlap{#1}}
\newcommand{\itab}[1]{\hspace{0em}\rlap{#1}}
\name{Константин Амеличев} % Your name
\address{(+7)9671295255 \\ kostya.amelichev@gmail.com} % Your phone number and email
\address{\href{http://github.com/kik0s}{github: KiK0S} \\ \href{http://codeforces.com/profile/kikos}{codeforces: KiKoS}}

\begin{document}

\begin{rSection}{Опыт работы}

{\bf Tinkoff.ru, Москва} \hfill {\bf Сентябрь 2020 - Настоящее время} \\ 
Младший backend-разработчик, занимаюсь сервисом предварительной обработки данных для контекстной рекламы.

{\bf Tinkoff Поколение, Москва} \hfill {\bf Сентябрь 2019 - Настоящее время} \\ 
Преподаватель курса алгоритмов и структур данных для школьников 9-11 классов.

{\bf С-инновации, Москва} \hfill {\bf Ноябрь 2019 - Декабрь 2019} \\ 
Заказной проект по прогнозированию результатов химических экспериментов на основе имеющихся экспериментальных данных.
\end{rSection}

\begin{rSection}{Образование}

{\bf Высшая Школа Экономики, Москва} \hfill {\bf Сентябрь 2019 - Июнь 2023} \\ 
Факультет компьютерных наук, ОП <<Прикладная математика и информатика>>.

\end{rSection}

\begin{rSection}{Учебные курсы}
{\bf \href{http://github.com/kik0s/tink}{\underline{Tinkoff Generation}}} \hfill {\bf Сентябрь 2018 - Июнь 2019} \\
Курсы для школьников по следующим направлениям: алгоритмы и структуры данных, машинное и глубокое обучение, олимпиадная математика.

{\bf \href{http://github.com/it-church/rbtrf-2018}{\underline{Робототехника}}} \hfill {\bf Сентябрь 2015 - Май 2019} \\
Курсы робототехники при школе 1540 (Arduino, dsPic). Разработка системы контроля и управления доступа в помещение, участие в международных и всероссийских соревнованиях
\href{https://wroboto.ru/rules/robotraffic/}{\underline{Роботраффик}}.

% {\bf Cisco IT Essentials} \hfill {\em Сентябрь 2016 - Май 2017} \\
% Курсы системного администрирования.

% {\bf Samsung IT School} \hfill {\em September 2016 - May 2017} \\
% Курсы Java и разработки Android-приложений.
\end{rSection}


\begin{rSection}{Проекты}
{\bf \href{http://github.com/it-church/neworgshop}{\underline{Карта московских магазинов органической продукции}}} \hfill {\bf Осень 2016 - Весна 2017} \\
Android приложение, показывающее пользователям сведения о ближайших магазинах, торгующих органической продукцией. Технологии: Java, Android SDK, Yandex map API.  \\
% {\bf \href{http://github.com/kik0s/spox}{\underline {Визуализатор логов}}} \\
% Целью проекта было сделать десктоп-приложение для отображения данных с датчкиков. Использовал Java, Swing.
{\bf \href{http://github.com/kik0s/dfvp}{\underline{Клиент-серверная игра}}}  \hfill {\bf Лето 2017} \\
Разработка простой компьютерной клиент-серверной игры. Технологии: Python, pygame. \\
{\bf \href{http://github.com/kik0s/codememes}{\underline {Codenames ИИ}}} \hfill {\bf Лето 2019} \\
Разработка телеграм-бота для игры в Codenames. Технологии: Python, gensim Word2Vec, python-telegram-bot. \\
\end{rSection}


\begin{rSection}{Умения}

\begin{tabular}{ @{} >{\bfseries}l @{\hspace{6ex}} l }
Языки программирования \ & Python, C++, Java\\
Фреймворки \ & Pandas, Numpy, Sklearn, Pytorch, python-telegram-bot, Flask\\
Сборка \ & Docker, CMake\\
Инструменты \ & Git, Linux \\
Языки \ & Английский B2
\end{tabular}

\end{rSection}

\begin{rSection}{Достижения \& Награды} 
    Победитель Всероссийской олимпиады школьников по информатике, 2019 \\
    Победитель Открытой олимпиады школьников по программированию, 2019 \\
    Бронзовая медаль Всероссийской командной олимпиады школьников по программированию, 2018
\end{rSection}

\end{document}
